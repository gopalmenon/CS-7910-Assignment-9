%%%%%%%%%%%%%%%%%%%%%%%%%%%%%%%%%%%%%%%%%
% Short Sectioned Assignment
% LaTeX Template
% Version 1.0 (5/5/12)
%
% This template has been downloaded from:
% http://www.LaTeXTemplates.com
%
% Original author:
% Frits Wenneker (http://www.howtotex.com)
%
% License:
% CC BY-NC-SA 3.0 (http://creativecommons.org/licenses/by-nc-sa/3.0/)
%
%%%%%%%%%%%%%%%%%%%%%%%%%%%%%%%%%%%%%%%%%

%----------------------------------------------------------------------------------------
%	PACKAGES AND OTHER DOCUMENT CONFIGURATIONS
%----------------------------------------------------------------------------------------

\documentclass[paper=a4, fontsize=11pt]{scrartcl} % A4 paper and 11pt font size

\usepackage[T1]{fontenc} % Use 8-bit encoding that has 256 glyphs
\usepackage{fourier} % Use the Adobe Utopia font for the document - comment this line to return to the LaTeX default
\usepackage[english]{babel} % English language/hyphenation
\usepackage{amsmath,amsfonts,amsthm} % Math packages

\usepackage{sectsty} % Allows customizing section commands
\usepackage[top=5em]{geometry}
\allsectionsfont{\centering \normalfont\scshape} % Make all sections centered, the default font and small caps

\usepackage{fancyhdr} % Custom headers and footers
\pagestyle{fancyplain} % Makes all pages in the document conform to the custom headers and footers
\fancyhead{} % No page header - if you want one, create it in the same way as the footers below
\fancyfoot[L]{} % Empty left footer
\fancyfoot[C]{} % Empty center footer
\fancyfoot[R]{\thepage} % Page numbering for right footer
\renewcommand{\headrulewidth}{0pt} % Remove header underlines
\renewcommand{\footrulewidth}{0pt} % Remove footer underlines
\setlength{\headheight}{5pt} % Customize the height of the header

\numberwithin{figure}{section} % Number figures within sections (i.e. 1.1, 1.2, 2.1, 2.2 instead of 1, 2, 3, 4)
\numberwithin{table}{section} % Number tables within sections (i.e. 1.1, 1.2, 2.1, 2.2 instead of 1, 2, 3, 4)

\setlength\parindent{0pt} % Removes all indentation from paragraphs - comment this line for an assignment with lots of text

\usepackage{mathtools}
\usepackage{amssymb}
\usepackage{gensymb}
\usepackage{chngcntr}
\usepackage{csquotes}
\usepackage{flexisym}
\usepackage{algorithm,algpseudocode}
\usepackage{tikz}

\usepackage{verbatim}
\usetikzlibrary{arrows,shapes}

\newcommand\Mycomb[2][n]{\prescript{#1\mkern-0.5mu}{}C_{#2}}

\counterwithout{figure}{section}
%----------------------------------------------------------------------------------------
%	TITLE SECTION
%----------------------------------------------------------------------------------------

\newcommand{\horrule}[1]{\rule{\linewidth}{#1}} % Create horizontal rule command with 1 argument of height

\title{	
\normalfont \normalsize 
\textsc{Utah State University, Computer Science Department} \\ [25pt] % Your university, school and/or department name(s)
\horrule{0.5pt} \\[0.4cm] % Thin top horizontal rule
\huge CS 7910 Computational Complexity\\Assignment 9 \\ % The assignment title
\horrule{2pt} \\[0.5cm] % Thick bottom horizontal rule
}

\author{Gopal Menon} % Your name

\date{\normalsize\today} % Today's date or a custom date

\begin{document}

\maketitle % Print the title

\begin{enumerate}
\item This exercise is for helping you to understand better the 2-approximation algorithm for the load balancing problem we studied in class.

You are asked to consult for a business where clients bring in jobs each day for processing. Each job $i$ has a processing time $t_i$ that is known when the job arrives. The company has a set of ten machines, and each job can be processed on any of these ten machines.

At the moment the business is running the simple 2-approximation greedy algorithm we discussed in class (i.e., the one without sorting the processing times; indeed, this is an \enquote{online} problem and you do not know the processing time of the next job until it arrives). They have been told that this may not be the best approximation algorithm possible, and they are wondering if they should be afraid of bad performance. However, they are reluctant to change the scheduling as they really like the simplicity of the current algorithm: jobs can be assigned to machines as soon as they arrive, without having to defer the decision until later jobs arrive.

In particular, they have heard that this algorithm can produce solutions with makespan as much as twice the minimum possible; but their experience with the algorithm has been quite good: They have been running it each day for the last month, and they have not observed it to produce a makespan more than 20 percent above the average load, i.e., $\frac{1}{10} \sum_i t_i$.

To try understanding why they do not seem to be encountering this factor-of-two behavior, you ask a bit about the kind of jobs and loads they see. You find out that the sizes of jobs range between 1 and 50, that is, $1 \leq t_i \leq 50$ for all jobs $i$; and the total load $\sum_i t_i$ is quite high each day: it is always at least 3000.

Prove that on the type of inputs the company sees, the greedy algorithm will always find a solution whose makespan is at most 20 percent above the average load (this implies that the approximation ratio of the algorithm on this special input pattern is no more than 1.2).

\end{enumerate}

%----------------------------------------------------------------------------------------

\end{document}